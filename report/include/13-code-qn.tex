\subsection{Quasi-Newton}
Código computacional para aproximar $\theta_1,~\theta_2,~\theta_3,~\theta_4$ Métodos Quasi-Newton.
Existem diversos Métodos Quasi-Newton, neste trabalho será utilizado o Método Quasi-Newton Broyden.



\begin{listing}[ht]
\caption{Método Quasi-Newton Broyden}
\label{alg:qn}
\inputminted
[
    frame=lines,
    framesep=2mm,
    baselinestretch=1.2,
    linenos
]
{python}{code/qn-broyden.py}
\end{listing}


\begin{listing}[ht]
\caption{Execução do Método Newton e Quasi-Newton Broyden}
\label{alg:main-qn}
\inputminted
[
    frame=lines,
    framesep=2mm,
    baselinestretch=1.2,
    linenos
]
{python}{code/main-sys.py}
\end{listing}
