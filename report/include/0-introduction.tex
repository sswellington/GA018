\section{Introdução}

% \subsection{Teoria geral para métodos iterativos}

% - Erro Absoluto e Relativo 

% \cite{decio2003}


% \subsection{Sistema de equação linear}
% \cite{decio2003} 

% \subsubsection{Eliminação de Gauss-Jordan}

% \subsubsection{Decomposição LU}
% \cite{gene2013}

% \subsubsection{Matriz Inversa}
% \cite{franco2006}

% \subsubsection{Matriz Transposta}
% \cite{franco2006}

% \subsubsection{Matriz Jacobiana}
% \cite{umetani1989jacobian_matrix}
% \cite{ruggiero2000calculo}

% \subsection{Matriz Hessiana}
% \cite{callahan2010hessian}

% \subsection{Norma matricial}
% \cite{gene2013}


% \subsection{Método de Newton-Raphson}
% \subsection{Quase-Newton}
% \subsection{Método de Newton-Raphson para Otimização}




As Bibliotecas utilizadas para o trabalhos são relacionadas ao projeto do Sympy\footnote{https://www.sympy.org/}.
O uso da biblioteca relacionado ao Sympy está representada pelo Algoritmo~\ref{alg:sympy} e os métodos para solução linear são implementados pelo Algoritmo~\ref{alg:metodos}, tendo por exemplo a Matriz Hessiana, a Matriz Jacobiana e a Eliminação de Gauss~\cite{franco2006,gene2013}.
Além disso, as classes Log e Error para auxiliar na implementação do métodos numéricos~\cite{decio2003}.
O primeiro, empregado para gerar o arquivo de log do algoritmo executado e informação do tempo de execução do método estudado, vide Algoritmo~\ref{alg:log}.
Por fim, usado para o cálculo de erro absoluto, relativo e também do cálculo da norma da matriz, vide o Algoritmo~\ref{alg:log}.
Portanto, o código fonte e os dados gerados pelos métodos estão disponível no Github~\footnote{https://github.com/sswellington/GA018}.


\begin{listing}[!h]
\caption{Biblioteca Sympy}
\label{alg:sympy}
\inputminted
[
    frame=lines,
    framesep=2mm,
    baselinestretch=1.2,
    linenos
]
{python}{code/sympy.py}
\end{listing}


\begin{listing}[!h]
\caption{Sistema Lineares}
\label{alg:metodos}
\inputminted
[
    frame=lines,
    framesep=2mm,
    baselinestretch=1.2,
    linenos
]
{python}{code/methods.py}
\end{listing}


\begin{listing}[!h]
\caption{Classe Log}
\label{alg:log}
\inputminted
[
    frame=lines,
    framesep=2mm,
    baselinestretch=1.2,
    linenos
]
{python}{code/Log.py}
\end{listing}


\begin{listing}[!h]
\caption{Classe Error}
\label{alg:error}
\inputminted
[
    frame=lines,
    framesep=2mm,
    baselinestretch=1.2,
    linenos
]
{python}{code/Error.py}
\end{listing}